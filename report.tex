\documentclass[a4paper,12pt,uplatex,dvipdfmx]{jsarticle}

% 数式
\usepackage{amsmath,amsfonts}
\usepackage{bm}
% グラフィック
\usepackage{graphicx}
\usepackage{tikz}

\usepackage{url}
\usetikzlibrary{intersections, calc, arrows}

\begin{document}

\title{Gamma関数とBeta関数について}
\author{@Tdrj2716}
\date{\today}
\maketitle

\section{Gamma関数}
\subsection{定義}

\subsection{階乗の一般化としての性質}

\section{Beta関数}
\subsection{定義}
\subsection{$x, y$ が自然数であるとき}
\subsection{Gamma関数との関係}

\begin{thebibliography}{9}
    \bibitem{mathtrain-gamma}
    高校数学の美しい物語, ガンマ関数(階乗の一般化)の定義と性質 \\
    \url{https://mathtrain.jp/gamma}
    \bibitem{mathtrain-beta}
    高校数学の美しい物語, ベータ関数の積分公式 \\
    \url{https://mathtrain.jp/beta}
    \bibitem{gamma-beta}
    倭算数理研究所, ガンマ関数とベータ関数のよくある関係 \\
    \url{https://wasan.hatenablog.com/entry/20110623/1308805478#%E3%82%AC%E3%83%B3%E3%83%9E%E9%96%A2%E6%95%B0%E3%81%A8%E3%83%99%E3%83%BC%E3%82%BF%E9%96%A2%E6%95%B0%E3%81%AE%E9%96%A2%E4%BF%82%E5%BC%8F}
\end{thebibliography}

\end{document}